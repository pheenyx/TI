\documentclass[10pt,a4paper]{scrartcl}
\PassOptionsToPackage{table}{xcolor}
\usepackage[utf8]{inputenc}
\usepackage[T1]{fontenc}
\usepackage[ngerman]{babel}
\usepackage{microtype, multicol, marginnote, bera, parskip}
\usepackage{listings, amsmath, amssymb, graphicx, tikz, epic}
\usepackage{stmaryrd} %for lightning arrow
\usepackage{pstricks, pst-node, pst-tree, pdflscape}
\usepackage[babel=true]{csquotes}
\usepackage{placeins}
\usepackage[labelformat=empty]{caption}
\tolerance=2000
\setcounter{secnumdepth}{0}
\usepackage[inner=2cm,outer=2cm,top=1.5cm,bottom=1.5cm,includeheadfoot]{geometry}
\usepackage{multirow}
\newcommand{\subExercise}[1]{\vspace{0.5em} \noindent{\bf #1)}}
\newcommand{\B}{\mathbb{B}}
\DeclareMathOperator{\op}{op}

\author{Michael Mardaus \and Andrey Tyukin}
\title{\includegraphics[scale=0.2]{../logo_schriftzug}\\
Technische Informatik: Abgabe 10}

\begin{document}

\maketitle

\section*{Exercise 10.1 (7-Segment display PLA)}

Truthtable for LED-display\\
\begin{tabular}{|l||l|l|l|l||l|l|l|l|l|l|l|}\hline
i  & $w$ & $x$ & $y$ & $z$ &     $a$ & $b$ & $c$ & $d$ & $e$ & $f$ & $g$ \\\hline\hline
0  & 0   & 0   & 0   & 0   &      1  &  1  &  1  &  1  &  1  &  1  &  0  \\\hline
1  & 0   & 0   & 0   & 1   &      0  &  1  &  1  &  0  &  0  &  0  &  0  \\\hline
2  & 0   & 0   & 1   & 0   &      1  &  1  &  0  &  1  &  1  &  0  &  1  \\\hline
3  & 0   & 0   & 1   & 1   &      1  &  1  &  1  &  1  &  0  &  0  &  1  \\\hline
4  & 0   & 1   & 0   & 0   &      0  &  1  &  1  &  0  &  0  &  1  &  1  \\\hline
5  & 0   & 1   & 0   & 1   &      1  &  0  &  1  &  1  &  0  &  1  &  1  \\\hline
6  & 0   & 1   & 1   & 0   &      0  &  0  &  1  &  1  &  1  &  1  &  1  \\\hline
7  & 0   & 1   & 1   & 1   &      1  &  1  &  1  &  0  &  0  &  0  &  0  \\\hline
8  & 1   & 0   & 0   & 0   &      1  &  1  &  1  &  1  &  1  &  1  &  1  \\\hline
9  & 1   & 0   & 0   & 1   &      1  &  1  &  1  &  0  &  0  &  1  &  1  \\\hline
10 & 1   & 0   & 1   & 0   &      -  &  -  &  -  &  -  &  -  &  -  &  -  \\\hline
11 & 1   & 0   & 1   & 1   &      -  &  -  &  -  &  -  &  -  &  -  &  -  \\\hline
12 & 1   & 1   & 0   & 0   &      -  &  -  &  -  &  -  &  -  &  -  &  -  \\\hline
13 & 1   & 1   & 0   & 1   &      -  &  -  &  -  &  -  &  -  &  -  &  -  \\\hline
14 & 1   & 1   & 1   & 0   &      -  &  -  &  -  &  -  &  -  &  -  &  -  \\\hline
15 & 1   & 1   & 1   & 1   &      -  &  -  &  -  &  -  &  -  &  -  &  -  \\\hline
\end{tabular}

This leads to these K-maps:\\
\begin{tabular}{|c||c|c|c|c|}
  \hline
 $a$     & \multicolumn{4}{c|}{$yz$} \\
  $wx$   & 00                 & 01                 & 11                 & 10                 \\ \hline\hline
  00     & \cellcolor{gray}1  & \cellcolor{gray}0  & \cellcolor{gray}1  & \cellcolor{gray}1  \\ \hline
  01     &                 0  &                 1  &                 1  &                 0  \\ \hline
  11     &                 d  &                 d  &                 d  &                 d  \\ \hline
  10     &                 1  &                 1  &                 d  &                 d  \\ \hline
\end{tabular}
\FloatBarrier

\end{document}

