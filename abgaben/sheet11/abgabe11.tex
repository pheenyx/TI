\documentclass[10pt,a4paper]{scrartcl}
\PassOptionsToPackage{table}{xcolor}
\usepackage[utf8]{inputenc}
\usepackage[T1]{fontenc}
\usepackage[ngerman]{babel}
\usepackage{microtype, multicol, marginnote, bera, parskip}
\usepackage{listings, amsmath, amssymb, graphicx, tikz, epic}
\usepackage{stmaryrd} %for lightning arrow
\usepackage{pstricks, pst-node, pst-tree, pdflscape}
\usepackage[babel=true]{csquotes}
\usepackage{placeins}
\usepackage[labelformat=empty]{caption}
\tolerance=2000
\setcounter{secnumdepth}{0}
\usepackage[inner=2cm,outer=2cm,top=1.5cm,bottom=1.5cm,includeheadfoot]{geometry}
\usepackage{multirow}
\usepackage{float}
\newcommand{\subExercise}[1]{\vspace{0.5em} \noindent{\bf #1)}}
\newcommand{\B}{\mathbb{B}}
\DeclareMathOperator{\op}{op}

\author{Michael Mardaus \and Andrey Tyukin}
\title{\includegraphics[scale=0.2]{../logo_schriftzug}\\
Technische Informatik: Abgabe 11}

\begin{document}

\maketitle

\section*{Exercise 11.1 (Johnny Simulator)}
\subExercise{a}
Program Listing:\\
\begin{tabular}{l|l|l}
Adr & Asm  & Op\\\hline
000 &  TAKE & 020\\
001 &  SUB  & 021\\
002 &  SAVE & 020\\
003 &  INC  & 022\\
004 &  TST  & 020\\
005 &  JMP  & 001\\
006 &  HLT  & 000\\
\end{tabular}

Puts $c = \lceil \frac{a}{b} \rceil$ (Adr 20 divided by Adr 21) into Adr 22.\\

\subExercise{b}
It takes $c$ times my loop (5 commands/cycles) plus 2 (TAKE and HLT) cycles.

\section*{Exercise 11.2 (Johnny Macro)}
The DBL macro would be a TAKE adr, ADD adr, SAVE adr combo.\\
The CLR macro would be a NULL adr, TAKE adr combo.

\section*{Excercise 11.4 (Caches)}
\subExercise{a}
In a RAM with 32 addresses and a cache with 5 lines address 13 could be in cache line 3 in a direct mapped cache. And in any of the 5 cachelines in a fullassociative cache, depending on the used cache mapping algorithm.\\
\subExercise{b}
5 cache lines is an unusual amount for a 32 adress RAM because 32 cannot be distributed evenly.
Powers of 2 are common cache sizes. But the number of cachelines should at least divide the number of addresses.

\end{document}

