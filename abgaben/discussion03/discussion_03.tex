\documentclass[10pt,a4paper]{scrartcl}
\usepackage[utf8]{inputenc}
\usepackage[T1]{fontenc}
\usepackage[ngerman]{babel}
\usepackage{microtype, multicol, marginnote, bera, parskip}
\usepackage{listings, amsmath, amssymb, graphicx, tikz, epic}
\usepackage{stmaryrd} %for lightning arrow
\usepackage{pstricks, pst-node, pst-tree, pdflscape}
\usepackage[babel=true]{csquotes}
\usepackage{placeins}
\tolerance=2000
\setcounter{secnumdepth}{0}
\usepackage[inner=2.5cm,outer=2.5cm,top=1.5cm,bottom=1.5cm,includeheadfoot]{geometry}
\newcommand{\subExercise}[1]{\vspace{1em} \noindent{\bf #1)}}
\renewcommand{\implies}{\rightarrow}
\title{\includegraphics[scale=0.2]{../logo_schriftzug}\\
Technische Informatik: Diskussion von 3.2}

\begin{document}

\maketitle
\section*{Discussion of the exercise 3.2 (Air conditioner)}
The description of the automatic part of the system contained four parts, all of which had to
be satisfied simultaneously (so there is an AND between each requirement).
\begin{itemize}
  \item Die Klimaanlage ist an, wenn der Temperatursensor an ist, ausser der Windsensor ist an und der
     Feuchtigkeitssensor ist aus.
  \item Die Klimaanlage ist auch an, wenn der Feuchtigkeitssensor an ist, ausser der Temperatursensor ist
     aus und der Windsensor ist an.
  \item Die Klimaanlage ist aus, wenn der Windsensor an ist, ausser der Temperatursensor und der
     Feuchtigkeitssensor sind beide gleichzeitig an.
  \item In allen anderen Faellen ist die Klimaanlage aus.
\end{itemize}
We used the following rule to translate the word ``ausser'' into boolean expressions: ``$x$ ausser $y$''
has been translated into ``$x \wedge \neg y$''. In the next step of the interpretation, we paraphrase 
sentences of type ``$b$, wenn $a$'' into ``Wenn $a$ dann $b$''. For variables $t$ (Temperature), $h$ (Humidity), $w$ (Wind), $a$ (Automatic part of the air conditioner control system) this gives us:
\begin{itemize}
  \item Wenn $t \wedge \neg (w \wedge \neg h) = t\bar w + t h =: e_1$ dann $a$.
  \item Wenn $h \wedge \neg (\neg t \wedge w) = h t + h \bar w =: e_2$ dann $a$.
  \item Wenn $w \wedge \neg (t \wedge h) = w \bar t + w \bar h =: e_3$ dann $\neg a$.
  \item Wenn $\neg e_1 \wedge \neg e_2 \wedge \neg e_3$ dann $\neg a$.
\end{itemize}
Now we transform ``Wenn ... dann ...'' sentences into logical implications:
\begin{itemize}
  \item $e_1 \implies a$
  \item $e_2 \implies a$
  \item $e_3 \implies \neg a$
  \item $\neg e_1 \wedge \neg e_2 \wedge \neg e_3 \implies \neg a$
\end{itemize}
We simplify the first two clauses using the following rule for the implications:
\[
  (a \implies z)\wedge(b \implies z) = (\bar a + z)(\bar b + z) = 
  \bar a \bar b + \bar a z + \bar b z + z = 
  \neg (a \vee b) + z = (a \vee b) \implies z
\]
into $e_{12} := th + t\bar w + h \bar w$ and obtain:
\begin{itemize}
  \item $e_{12} \implies a$
  \item $e_3 \implies \neg a$
  \item $\neg e_{12} \wedge \neg e_3 \implies \neg a$
\end{itemize}
We now make the observation that $e_{12}$ and $e_3$ are never both $1$ at the same time: 
$e_{12} \wedge e_3 = 0$ (pointwise, as functions $\mathbb{B}^3 \to \mathbb{B}$).
Let us for a moment call such functions \emph{nonconflicting}.
That means that even if the implications of $e_{12}$ and $e_3$ are opposite, 
we still can construct a function $a$ which fulfills all these implications:

\vspace{0.5em}
\begin{tabular}{|ccc|c|c|}
\hline
$h$ & $t$ & $w$ & $e_{12}$ & $e_3$ \\
\hline
0 & 0 & 0 & 0 & 0 \\ 
0 & 0 & 1 & 0 & 1 \\ 
0 & 1 & 0 & 1 & 0 \\  
0 & 1 & 1 & 0 & 1 \\ 
1 & 0 & 0 & 1 & 0 \\ 
1 & 0 & 1 & 0 & 1 \\ 
1 & 1 & 0 & 1 & 0 \\ 
1 & 1 & 1 & 1 & 0 \\ 
\hline
\end{tabular}

In particular, this means that there is at least no ambiguity about the precedence
of the rules: the rules do not conflict, and therefore their order is irrelevant.
How can we combine these implications into one single function $a$?

\noindent{\bf{Lemma}} 
Let $x,y,a: \mathbb{B}^n \to \mathbb{B}$ be some boolean functions with
$x,y$ nonconflicting (i.e. $x\wedge y = 0$) and 
\[
  (x \implies a) \wedge (y \implies \neg a) \wedge ((\neg x \wedge \neg y) \implies \neg a).
\]

\noindent{\bf{Claim:}} a = x

\noindent{\bf{Proof:}} We show that $(x \implies a)$ and $(a \implies x)$, from which the
equality easily follows: 
\[
  (a \implies x) \wedge (x \implies a) = (\bar a + x)(\bar x + a) = \bar a \bar x + 0 + 0 + a x = 
  a \leftrightarrow x.
\]
The $(x \implies a)$ clause is already assumed, there is nothing to show.
For $(a \implies x)$ we turn the two other assumed applications around, 
negating the arguments (using de Morgan in second case):
\begin{align*}
  a \implies \neg y \\
  a \implies (x \vee y)
\end{align*}
Now we can combine the right sides of these two implications:
\[
  \bar y (x + y) = \bar y x = \bar y x + 0 
  \overset{\textrm{nonconflicting}}=
  \bar y x + x y = x
\]
obtaining $(a \implies x)$. Thus the equality holds.

With this lemma, 
it is immediately obvious that the implementation of the automatic part
of the air conditioner is just $e_{12} = t h + t\bar w + h \bar w$, the
$e_3$ OFF-part can be ignored alltogether, 
at least as long as it is nonconflicting with the ON-part.

Notice the $\bar y x = x\wedge\neg y$ expression in the last line of the proof.
This expression occurred in our original formulation, 
back then we simply did not notice that the conjunction with the
$\bar y$ from the $(y \implies \neg a)$ requirement has no effect, and calculated 
everything by brute force\footnote[1]{
  Unfortunately, we used a ``\textbackslash not'' instead of a ``\textbackslash neg'' in our TeX, 
  this is why there was a cancelled parenthesis instead of a negation sign in the first line of
  the calculation.
}:

\begin{align*}
a &= 
  ((t \wedge \neg (w \wedge \neg h)) \vee (h \wedge \neg (\neg t \wedge w))) 
  \wedge 
  \not (w \wedge \neg (t \wedge h)) \\
  &\overset{De Morgan}=
  ((t \wedge (\neg w \vee h)) \vee (h \wedge (t \vee \neg w))) \wedge (\neg w \vee (t \wedge h)) \\
  &\overset{distr, idempot}=
  (t\bar w + th + ht + h \bar w)(\bar w + th) \\
  &\overset{distr, assoc}=
  t\bar w + th\bar w + h\bar w + th\bar w + th + th \bar w \\
  &\overset{absorp}=
  t\bar w + th + h\bar w
\end{align*}

This is exactly the function represented by the table:

\vspace{0.5 em}
\begin{tabular}{|ccc|c|}
\hline
$t$ & $h$ & $w$ & $e_{12}$ \\
\hline
0 & 0 & 0 & 0 \\ 
0 & 0 & 1 & 0 \\ 
0 & 1 & 0 & 1 \\  
0 & 1 & 1 & 0 \\ 
1 & 0 & 0 & 1 \\ 
1 & 0 & 1 & 0 \\ 
1 & 1 & 0 & 1 \\ 
1 & 1 & 1 & 1 \\ 
\hline
\end{tabular}

However, if we replace the AND by OR in $e_{12} \wedge \neg e_3$, we obtain a different function:

\vspace{0.5 em}
\begin{tabular}{|ccc|c|}
\hline
$t$ & $h$ & $w$ & $e_3\vee \neg e_3 = th + \bar w$ \\
\hline
0 & 0 & 0 & 1 \\ 
0 & 0 & 1 & 0 \\ 
0 & 1 & 0 & 1 \\  
0 & 1 & 1 & 0 \\ 
1 & 0 & 0 & 1 \\ 
1 & 0 & 1 & 0 \\ 
1 & 1 & 0 & 1 \\ 
1 & 1 & 1 & 1 \\ 
\hline
\end{tabular}

The difference is in the $0$-th row: it's exactly the case where none of the rules 1-3 matches.
By implementing it with an OR, we change the default from OFF to ON, in contradiction to the
fourth requirement about the default state. Therefore, we cannot just flip AND to OR.

\section*{Cancellation of non-epimorphisms.}
A word of caution on cancelling non-surjective\footnote{It was of course \emph{surjective}, not \emph{injective}, always mix them up, sorry.} functions. 
Let $f_1, f_2, f_3: \mathbb{B}^3 \to \mathbb{B}$ be some boolean functions. 
From the unequality of $\phi_\wedge: (A,B,C) \mapsto (A \vee B) \wedge C$ and 
$\phi_\vee: (A,B,C) \mapsto (A \vee B) \vee C$ it does \emph{not} follow that 
\[
  \phi_\vee(f_1, f_2, f_3) \neq \phi_\wedge(f_1, f_2, f_3).
\]
One still has to look carefully at the function $f$.

Example from the first exercise sheet, using the consensus theorem:
$f_1(x,y,z) = x y$, $f_2(x,y,z) = \bar x z$, $f_3(x,y,z) = x y + \bar x z + y z$.
Althought $\phi_\vee \neq \phi_\wedge$, it still holds: 
$\phi_\vee \circ f = \phi_\wedge \circ f$. 
This happens every time when the range of $f$ and the 
set $\{x: \phi_\wedge(x) \neq \phi_\vee(x)\}$ are disjoint. 
The function $f$ just happens not to take any values 
where $\phi_\vee$ and $\phi_\wedge$ differ:

\vspace{0.5 em}
\begin{tabular}{|ccc|ccc|c|c|c|c|}
\hline
$x$ & $y$ & $z$ & $f_1$ & $f_2$ & $f_3$ & $\phi_\wedge$ & $\phi_\vee$ & $\phi_\wedge \circ f$ & $\phi_\vee \circ f$ \\
\hline
0 & 0 & 0 & 0 & 0 & 0 & 0 & 0  & 0 & 0 \\
0 & 0 & 1 & 0 & 1 & 1 & 0 & 1  & 1 & 1 \\
0 & 1 & 0 & 0 & 0 & 0 & 0 & 1  & 0 & 0 \\
0 & 1 & 1 & 0 & 1 & 1 & 1 & 1  & 1 & 1 \\
1 & 0 & 0 & 0 & 0 & 0 & 0 & 1  & 0 & 0 \\
1 & 0 & 1 & 0 & 0 & 0 & 1 & 1  & 0 & 0 \\
1 & 1 & 0 & 1 & 0 & 1 & 0 & 1  & 1 & 1 \\
1 & 1 & 1 & 1 & 0 & 1 & 1 & 1  & 1 & 1 \\
\hline
\end{tabular}

The range of $f$ is $\{000, 011, 101\}$, but $\phi_\wedge$ and $\phi_\vee$ differ only on
$\{001, 010, 100, 110\}$.

All this, as it turned out, had no connection to the actual problem, because in the case of the 
air conditioner flipping AND to OR yielded a different function.
\end{document}
