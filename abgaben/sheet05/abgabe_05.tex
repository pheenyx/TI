\documentclass[10pt,a4paper]{scrartcl}
\PassOptionsToPackage{table}{xcolor}
\usepackage[utf8]{inputenc}
\usepackage[T1]{fontenc}
\usepackage[ngerman]{babel}
\usepackage{microtype, multicol, marginnote, bera, parskip}
\usepackage{listings, amsmath, amssymb, graphicx, tikz, epic}
\usepackage{stmaryrd} %for lightning arrow
\usepackage{pstricks, pst-node, pst-tree, pdflscape}
\usepackage[babel=true]{csquotes}
\usepackage{placeins}
\tolerance=2000
\setcounter{secnumdepth}{0}
\usepackage[inner=2cm,outer=2cm,top=1.5cm,bottom=1.5cm,includeheadfoot]{geometry}
\usepackage{multirow}
\newcommand{\subExercise}[1]{\vspace{0.5em} \noindent{\bf #1)}}
\author{Michael Mardaus \and Andrey Tyukin}
\title{\includegraphics[scale=0.2]{../logo_schriftzug}\\
Technische Informatik: Abgabe 5}

\begin{document}

\maketitle

\section*{Exercise 5.1 (Quine McCluskey and K-maps)}
\subExercise{a} We apply the Quine McCluskey algorithm to obtain optimal disjunctive representation of $f$.

\vspace{0.5em}
\begin{tabular}{|l||l|l|l|l|l|l|l|l|l|}\hline
\#(0) & decimal & $x_1x_2x_3x_4x_5$ & \multicolumn{7}{c|}{Reduction-Links}                \\\hline\hline
1     & 15      & 01111             & &&&&&&                                   \\
      & 30      & 11110             & \textcolor{red}{$\star^1$} & \textcolor{orange}{$\star^2$} &&&&&              \\
      & 29      & 11101             & &&\textcolor{cyan}{$\star^3$}&$\star^4$&$\star^5$&&        \\\hline
2     & 28      & 11100             & \textcolor{red}{$\star^1$}&&\textcolor{cyan}{$\star^3$}&&&$\star^6$&        \\
      & 25      & 11001             & &&\textcolor{gray}{$\star^8$}&$\star^4$&&&$\star^7$        \\
      & 22      & 10110             & &\textcolor{orange}{$\star^2$}&&&&&                          \\
      & 21      & 10101             & \textcolor{green}{$\star^9$}&&&&$\star^5$&&                 \\\hline
3     & 24      & 11000             & &$\star^{10}$&&&&$\star^6$&$\star^7$     \\
      & 17      & 10001             & \textcolor{green}{$\star^9$}&&\textcolor{gray}{$\star^8$}&&&&                 \\
      & 9       & 01001             & &&\textcolor{violet}{$\star^{11}$}&&&&                       \\\hline
4     & 8       & 01000             & \textcolor{blue}{$\star^{12}$}&$\star^{10}$&\textcolor{violet}{$\star^{11}$}&&&&\\
      & 2       & 00010             & &\textcolor{magenta}{$\star^{13}$}&&&&&                       \\\hline
5     & 0       & 00000             & \textcolor{blue}{$\star^{12}$}&\textcolor{magenta}{$\star^{13}$}&&&&&           \\\hline
\end{tabular}\\
\hspace*{3cm}$\Downarrow$\hspace*{3cm}$\Downarrow$\\
\hspace*{3.5mm}
\begin{tabular}{|l||l|l|l|l|l|l|}\hline
\#(0) & decimal & $x_1x_2x_3x_4x_5$ & \multicolumn{4}{c|}{Reduction-Links}                \\\hline\hline
1     & 15      & 01111             & &&&                                   \\
      & 28,30   & 111*0             & &&&              \\
      & 22,30   & 1*110             & &&&        \\
      & 28,29   & 1110*             & \textcolor{red}{$\star^1$}&&&        \\
      & 25,29   & 11*01             & &\textcolor{green}{$\star^2$}&\textcolor{blue}{$\star^3$}&        \\
      & 21,29   & 1*101             & &&&\textcolor{orange}{$\star^4$}                          \\\hline
2     & 24,28   & 11*00             & &\textcolor{green}{$\star^2$}&&                \\
      & 24,25   & 1100*             & \textcolor{red}{$\star^1$}&&&     \\
      & 17,25   & 1*001             & &&&\textcolor{orange}{$\star^4$}                 \\
      & 17,21   & 10*01             & &&\textcolor{blue}{$\star^3$}&                    \\\hline
3     & 8,24    & *1000             & &&&\\
      & 8,9     & 0100*             & &&&                       \\\hline
4     & 0,8     & 0*000             & &&&           \\
      & 0,2     & 000*0             & &&&           \\\hline
\end{tabular}\\
\hspace*{3cm}$\Downarrow$\hspace*{3cm}$\Downarrow$\\
\hspace*{3.5mm}
\begin{tabular}{|l||l|l|}\hline
\#(0) & decimal     & $x_1x_2x_3x_4x_5$    \\\hline\hline
1     & 15          & 01111                \\
      & 28,30       & 111*0                \\
      & 22,30       & 1*110                \\
      & 24,25,28,29 & 11*0*                \\
      & 17,21,25,29 & 1**01                \\\hline
3     & 8,24        & *1000                \\
      & 8,9         & 0100*                \\\hline
4     & 0,2         & 000*0                \\\hline
\end{tabular}\\
\hspace*{3cm}$\Downarrow$\hspace*{3cm}$\Downarrow$\\
\hspace*{3.5mm}
\begin{tabular}{|l||l|l|l|l|l|l|l|l|l|l|l|l|l|l|}\hline
Primimplicant & 0 & 2 & \cellcolor{gray}8 & 9 & 15 & 17 & \cellcolor{gray}21 & 22 & 24 & \cellcolor{gray}25 & 28 & \cellcolor{gray}29 & \cellcolor{gray}30 &  \\\hline\hline
 01111        &   &   & \cellcolor{gray}  &   & 1  &    & \cellcolor{gray}   &    &    & \cellcolor{gray}   &    & \cellcolor{gray}   & \cellcolor{gray}   & \\\hline
 \cellcolor{gray}111*0        & \cellcolor{gray}  & \cellcolor{gray}  & \cellcolor{gray}  & \cellcolor{gray}  & \cellcolor{gray}   & \cellcolor{gray}   & \cellcolor{gray}   & \cellcolor{gray}   &  \cellcolor{gray}  & \cellcolor{gray}   & \cellcolor{gray}1  & \cellcolor{gray}   & \cellcolor{gray}1  & reduced r 4\\ \hline
 1*110        &   &   & \cellcolor{gray}  &   &    &    & \cellcolor{gray}   & 1  &    & \cellcolor{gray}   &    & \cellcolor{gray}   & \cellcolor{gray}1  & \\
 11*0*        &   &   & \cellcolor{gray}  &   &    &    & \cellcolor{gray}   &    & 1  & \cellcolor{gray}1  & 1  & \cellcolor{gray}1  & \cellcolor{gray}   & \\
 1**01        &   &   & \cellcolor{gray}  &   &    & 1  & \cellcolor{gray}1  &    &    & \cellcolor{gray}1  &    & \cellcolor{gray}1  & \cellcolor{gray}   & \\  \hline
 \cellcolor{gray}*1000        & \cellcolor{gray}  & \cellcolor{gray}  & \cellcolor{gray}1 &  \cellcolor{gray} & \cellcolor{gray}   & \cellcolor{gray}   & \cellcolor{gray}   & \cellcolor{gray}   & \cellcolor{gray}1  & \cellcolor{gray}   & \cellcolor{gray}   & \cellcolor{gray}   & \cellcolor{gray}   & reduced r 4\\  \hline
 0100*        &   &   & \cellcolor{gray}1 & 1 &    &    & \cellcolor{gray}   &    &    & \cellcolor{gray}   &    & \cellcolor{gray}   & \cellcolor{gray}   & \\
 000*0        & 1 & 1 & \cellcolor{gray}  &   &    &    & \cellcolor{gray}   &    &    & \cellcolor{gray}   &    & \cellcolor{gray}   & \cellcolor{gray}   & \\ \hline
 reduced b/c  &   &   & c9                &   &    &    & c17                &    &    & c17                &    & c17                & c22                & \\ \hline
\end{tabular}

...here lives the big table for row and column cheating...\\

$\Longrightarrow f = \neg x_1x_2\neg x_3\neg x_4+
                     \neg x_1\neg x_2\neg x_3\neg x_5+
                     \neg x_1x_2x_3x_4x_5+
                     x_1x_2\neg x_4+
                     x_1\neg x_4x_5+
                     x_1x_3x_4\neg x_5$\\


\subExercise{b}
Use K-Maps to find a cost-minimal DNF.\\
\begin{tabular}{|c||c|c|c|c|}
  \hline
  \multicolumn{5}{|c|}{$x_1=0$} \\ \hline
            & \multicolumn{4}{c|}{$x_4x_5$} \\
$x_2x_3$ & 00                  & 01                  & 11                 & 10                  \\ \hline\hline
    00   & \cellcolor{orange}1 &                     &                    & \cellcolor{orange}1 \\ \hline
    01   &                     &                     &                    &                     \\ \hline
    11   &                     &                     & 1                  &                     \\ \hline
    10   & \cellcolor{blue}1   & \cellcolor{blue}1   &                    &                     \\
  \hline
\end{tabular}
\begin{tabular}{|c||c|c|c|c|}
  \hline
  \multicolumn{5}{|c|}{$x_1=1$} \\ \hline
            & \multicolumn{4}{c|}{$x_4x_5$} \\
$x_2x_3$ & 00                  & 01                  & 11                 & 10                  \\ \hline\hline
    00   &                     & \cellcolor{green}1  &                    &                     \\ \hline
    01   &                     & \cellcolor{green}1  &                    & \cellcolor{magenta}1\\ \hline
    11   & \cellcolor{red}1    & \cellcolor{yellow}1 &                    & \cellcolor{magenta}1\\ \hline
    10   & \cellcolor{red}1    & \cellcolor{yellow}1 &                    &                     \\
  \hline
\end{tabular}\\[5mm]
which yields: $f   =  \textcolor{blue}{\neg x_1x_2\neg x_3\neg x_4}+
                      \textcolor{orange}{\neg x_1\neg x_2\neg x_3\neg x_5}+
                      \neg x_1x_2x_3x_4x_5+
                      \textcolor{red}{x_1x_2\neg x_4}+
                      \textcolor{green}{x_1\neg x_4x_5}+
                      \textcolor{magenta}{x_1x_3x_4\neg x_5}$\\
The same result as in part a).


\FloatBarrier
\newpage
\section*{Exercise 5.2 (Row and Column-Rules are not a function)}
\begin{tabular}{|c||c|c|c|}
  \hline
Primimplic.       & \cellcolor{gray}$C_1=0$  & $C_2=1$   & $C_3=2$           \\ \hline\hline
$R_1 = 01$     & \cellcolor{gray}         & 1         &                   \\ \hline
$R_2 = 0\ast$  & \cellcolor{gray}1        & 1         &                   \\ \hline
$R_3 = \ast 0$ & \cellcolor{gray}1        &           & 1                 \\
  \hline
\end{tabular}\\
We deleted Column $C_1$ with the Column-Rule, because $C_3 \leq C_1$ and overlaps it. Now we can either use $R_3+R_2 = \neg x_2+\neg x_1$ or $R_3+R_1 = \neg x_2 + \neg x_1x_2$
as our minimal solution.

\FloatBarrier
\section*{Exercise 5.3 (Greedy Algorithm)}
We consider boolean matrices with at least one $1$ in each column and 
at most two $1$'s in each row. 
The greedy algorithm repeatedly traverses all rows
and in each step chooses the row that covers the most columns, until all
colums are covered.
 We assume top-down traversal order, furthermore we assume that if two rows
cover the same number of columns, that the one with smaller index is chosen.

\subExercise{a} {\textbf{Claim:}} The greedy algorithm generally does not find the
optimal solution.

\noindent {\textbf{Proof:}} Consider the following example:

\vspace{0.5em}
\begin{tabular}{|c c c c|}
\hline
1 & 1 & 0 & 0 \\
1 & 0 & 1 & 0 \\
0 & 1 & 0 & 1 \\
\hline
\end{tabular}

In the first iteration each row covers two columns, therefore the algorithm would just pick the first row.
In subsequent steps, it also has to take second and third row into the cover, because both are necessary (there is only one $1$ in third and fourth column). However, the optimal solution does not require all rows: the second and third row alone would be sufficient.

\subExercise{b} Let $n$ be the number of rows and $m$ be the number of columns of the matrix (the restrictions on the numbers of $1$'s in rows and column shall still hold).

{\textbf{Claim:}} There exists a cover with $m - k \leq m$ rows iff there exists 
$K\subset\{1,\dots,n\}$ of cardinality $k$ such that rows $\{z_j\}_{j\in K}$ cover $2 k$ columns.

\noindent {\textbf{Proof:}} First, let there be such a subset $K$. By assumption, $\{z_j\}_{j\in K}$ 
already cover $2 k$ columns, and $m - 2 k$ columns remain uncovered. Because there is at least one $1$ in each of the remaining $m - 2 k$ columns, we need at most $l \leq m - 2 k$ more rows to cover these columns. Together, we need $k + l \leq k + (m - 2 k) = m - k$ rows to cover all columns.

For the opposite direction, let $C \subset \{1,\dots,n\}$ with $|C| = m - k$ be a subset of row-indices which cover all $m$ columns. 
Let $\hat K \subset C$ be a maximal set with the property that $\{z_j\}_{j\in \hat K}$ cover $2 |\hat K|$ rows. 
Because of the maximality and the restriction that there are at most two $1$'s per row, adding further rows with indices from $C - \hat K$ to $\hat K$ increases the number of covered columns at most by $1$ per added row, that is, 
we can cover at most $2 |\hat K| + (|C| - |\hat K|) = |\hat K| + |C| = |\hat K| + m - k$ rows. 
If $|\hat K| < k$ would hold, then we could cover at most $m - 1$ columns, which contradicts the assumption that rows with indices from $C$ cover \emph{all} columns. Therefore $|\hat K| \geq k$ must hold. In particular, there exists a subset $K \subset \hat K$ of size exactly $k$ such that $\{z_j\}_{j\in K}$ cover exactly $2 k$ columns.
\end{document}
