\documentclass[10pt,a4paper]{scrartcl}
\PassOptionsToPackage{table}{xcolor}
\usepackage[utf8]{inputenc}
\usepackage[T1]{fontenc}
\usepackage[ngerman]{babel}
\usepackage{microtype, multicol, marginnote, bera, parskip}
\usepackage{listings, amsmath, amssymb, graphicx, tikz, epic}
\usepackage{stmaryrd} %for lightning arrow
\usepackage{pstricks, pst-node, pst-tree, pdflscape}
\usepackage[babel=true]{csquotes}
\usepackage{placeins}
\usepackage[labelformat=empty]{caption}
\tolerance=2000
\setcounter{secnumdepth}{0}
\usepackage[inner=2cm,outer=2cm,top=1.5cm,bottom=1.5cm,includeheadfoot]{geometry}
\usepackage{multirow}
\newcommand{\subExercise}[1]{\vspace{0.5em} \noindent{\bf #1)}}
\newcommand{\B}{\mathbb{B}}
\DeclareMathOperator{\op}{op}

\author{Michael Mardaus \and Andrey Tyukin}
\title{\includegraphics[scale=0.2]{../logo_schriftzug}\\
Technische Informatik: Abgabe 9}

\begin{document}

\maketitle

\section*{Exercise 9.1 (Adding)}


\subExercise{a}
20+15
\begin{tabular}{|c|c|c|c|c|}
  \hline
dec& $bin$   & $K_2(bin)$  \\\hline
20 & 010100  & 010100 \\          
15 & 001111  & 001111 \\
\hline\hline
(35) & 100011 & 011101 = -29 \\\hline
\end{tabular}

\subExercise{b}
-13+7
\begin{tabular}{|c|c|c|c|c|}
  \hline
dec& $bin$   & $K_2(bin)$  \\\hline
-13 & 001101 & 110011 \\          
7 & 000111   & 000111 \\
\hline\hline
(-6) & 111010 & 000110 = -6 \\\hline
\end{tabular}

\subExercise{c}
11-28
\begin{tabular}{|c|c|c|c|c|}
  \hline
dec& $bin$   & $K_2(bin)$  \\\hline
11 & 001011 & 001011 \\          
-28 & 011100 & 100100 \\
\hline\hline
(-17) & 101111 & 010001 = -17 \\\hline
\end{tabular}

\subExercise{d}
-13+7
\begin{tabular}{|c|c|c|c|c|}
  \hline
dec& $bin$   & $K_2(bin)$  \\\hline
-19 & 010011 & 101101 \\          
-22 & 010110 & 101010 \\
\hline\hline
(-41) & (1)010111 & 101001 = 25 \\\hline
\end{tabular}
\FloatBarrier

\section*{Exercise 9.3 + 9.4 (Number representations)}

\subExercise{a}
To perform the actual calculation, we transform everything into the binary representation:
\begin{align*}
(8.125)_{10} &= (1000.001)_2 \\
(B3.09)_{16} &= (10110011.00001001)_2 \\
(27.65)_8 &= (10111.110101)_2
\end{align*}

Then we just calculate, step by step (everything binary if not stated otherwise):
\begin{align*}
1000.001 - 111.011 &= 0.11 \\
0.11 + 10110011.00001001 &= 10110011.11001001 \\
10110011.11001001 - 10111.110101 &= 10011011.11110101 = (155.95703125)_{10}
\end{align*}
From the last binary representation we see that the representation in required floating point format is
$+(0.9BF5)_{16}\cdot16^2$. Unfortunately, not all the bits required to represent the number fit into mantissa:
\begin{align*}
0 \quad 1001101111 \quad 00010
\end{align*}
The resulting error is $(0.00110101)_2 = 3/16 + 1/64 + 1/256 = (0.2070325)_{10}$.

\subExercise{b} (omitted)
% Again, everything to binary first:
% \begin{align*}
% (A3.12)_{16} &= 10100011.0001001 \\
% (10.7041)_8 &= 1000.111000100001 \\
% (17.34)_{10} &= 10001.0101011100001... \\
% \end{align*}
% Calculating:

\end{document}
