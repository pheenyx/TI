\documentclass[10pt,a4paper]{scrartcl}
\usepackage[utf8]{inputenc}
\usepackage[T1]{fontenc}
\usepackage[ngerman]{babel}
\usepackage{microtype, multicol, marginnote, bera, parskip}
\usepackage{listings, amsmath, amssymb, graphicx, tikz, epic}
\usepackage{stmaryrd} %for lightning arrow
\usepackage{pstricks, pst-node, pst-tree, pdflscape}
\usepackage[babel=true]{csquotes}
\tolerance=2000
\setcounter{secnumdepth}{0}
\usepackage[inner=2.5cm,outer=2.5cm,top=1.5cm,bottom=1.5cm,includeheadfoot]{geometry}

\author{Michael Mardaus}
\title{\includegraphics[scale=0.2]{logo_schriftzug}\\ Technische Informatik:
Abgabe 1}

\begin{document}

\maketitle

\section{Aufgabe 1}

\begin{tabular}{|l|l|l|l|}\hline
Dualsystem & Oktalsystem & Dezimalsystem & Hexadezimalsystem \\\hline\hline
\textbf{101110101101} & 5655 & 2989 & 0B AD \\\hline
11001000101 & \textbf{3105} & 1605 & 06 45 \\\hline
1110001111 & 1617 & \textbf{911}$\ast$ & 03 8F \\\hline
111110101100 & 7654 & 4012 & \textbf{0F AC} \\\hline
\end{tabular}
\[\ast 911 = 512 + 256 + 128 + 8 + 4 + 2 + 1\] 
 

\section{Aufgabe 2}
Beweisen oder widerlegen Sie: $(a.b) + (\neg a.c) = (a.b) + (\neg a.c) + (b.c)$

\begin{eqnarray*}
& (a.b) + (\neg a.c) + (b.c) &\\
= & (a.b) + (\neg a.c) + (b.c) + (a.\neg a) & \text{false addiert}\\
= & (a.b) + (\neg a.c) + (a.b.c) + (\neg a.b.c) & \text{Distributivgesetz}\\
= & (a.b)  + (a.b.c)+ (\neg a.c) + (\neg a.c.b) & \text{Kommutativgesetz}\\
= & (a.b).(1 + c) + (\neg a.c).(1 + b) & \text{Ausklammern, Distributivgesetz}\\
= & (a.b) + (\neg a.c) & \text{true or \_ = true, weglassen in Produkt}\\
\end{eqnarray*}


\section{Aufgabe 3}
\subsection{a)}
\begin{tabular}{|l||l|l|l||l|}\hline
i & $x_1$ & $x_2$ & $x_3$ & $f(x_1,x_2,x_3)$ \\\hline\hline
0 & 0 & 0 & 0 & 0 \\\hline
1 & 0 & 0 & 1 & 0 \\\hline
2 & 0 & 1 & 0 & 0 \\\hline
3 & 0 & 1 & 1 & 0 \\\hline
4 & 1 & 0 & 0 & 0 \\\hline
5 & 1 & 0 & 1 & 1 \\\hline
6 & 1 & 1 & 0 & 1 \\\hline
7 & 1 & 1 & 1 & 1 \\\hline
\end{tabular}

\paragraph{DNF}
DNF = Minterme der einschlägigen Indizes (Sum of Products)\\
$m_5 + m_6 + m_7 = x_1.\neg x_2.\neg x_3 + x_1.x_2.\neg x_3 + x_1.x_2.x_3$

\paragraph{KNF}
KNF = Maxterme (=$\neg$ Minterme) der nullschlägigen Indizes (Product of Sums)\\
$M_0 . M_1 . M_2 . M_3 . M_4 = (x_1+x_2+x_3) . (x_1+x_2+\neg x_3) . (x_1+\neg x_2+x_3) . (x_1+\neg x_2+\neg x_3) . (\neg x_1+x_2+x_3)$

\subsection{b)}
\begin{tabular}{|l||l|l|l||l|}\hline
i & $x_1$ & $x_2$ & $x_3$ & $f(x_1,x_2,x_3)$ \\\hline\hline
0 & 0 & 0 & 0 & 0 \\\hline
1 & 0 & 0 & 1 & 1 \\\hline
2 & 0 & 1 & 0 & 0 \\\hline
3 & 0 & 1 & 1 & 1 \\\hline
4 & 1 & 0 & 0 & 1 \\\hline
5 & 1 & 0 & 1 & 0 \\\hline
6 & 1 & 1 & 0 & 1 \\\hline
7 & 1 & 1 & 1 & 0 \\\hline
\end{tabular}

\paragraph{DNF}
DNF = Minterme der einschlägigen Indizes (Sum of Products)\\
$m_1 + m_3 + m_4 + m_6 = \neg x_1.\neg x_2.x_3 + \neg x_1.x_2.x_3 + x_1.\neg x_2.\neg x_3 + x_1.x_2.\neg x_3$

\paragraph{KNF}
KNF = Maxterme (=$\neg$ Minterme) der nullschlägigen Indizes (Product of Sums)\\
$M_0 . M_2 . M_5 . M_7  = (x_1+x_2+x_3) . (x_1+\neg x_2+x_3) . (\neg x_1+x_2+\neg x_3) . (\neg x_1+\neg x_2+\neg x_3)$


\section{Aufgabe 4}
Definition: \\
a = Fr Hoffman kommt, b = Hr Hoffman kommt, c = Fr. Beck kommt, d = Hr Reuschenbach kommt\\
Bekannt: $x \Rightarrow y \Longleftrightarrow \neg x + y$\\
Aussagen: 
\begin{enumerate}
 \item $a \Rightarrow b \Longleftrightarrow \neg a + b$ 
 \item $\neg c \Rightarrow d \Longleftrightarrow c + d$
 \item $d \Rightarrow b \Longleftrightarrow \neg d + b$
 \item $(a.c) \Rightarrow \neg d \Longleftrightarrow (\neg a+\neg c) + \neg d \Longleftrightarrow \neg (a.c.d)$
\end{enumerate}

\[\Longrightarrow f(a,b,c,d)= (\neg a+b).(c+d).(\neg d+b).(\neg (a.c.d))\]
\begin{tabular}{|l||l|l|l|l||l|}\hline
i & $a$ & $b$ & $c$ & $d$ & $f(a,b,c,d)$ \\\hline\hline
0 & 0 & 0 & 0 & 0 & 0 \\\hline
1 & 0 & 0 & 0 & 1 & 0 \\\hline
2 & 0 & 0 & 1 & 0 & 1 \\\hline
3 & 0 & 0 & 1 & 1 & 0 \\\hline
4 & 0 & 1 & 0 & 0 & 0 \\\hline
5 & 0 & 1 & 0 & 1 & 1 \\\hline
6 & 0 & 1 & 1 & 0 & 1 \\\hline
7 & 0 & 1 & 1 & 1 & 1 \\\hline
8 & 1 & 0 & 0 & 0 & 0 \\\hline
9 & 1 & 0 & 0 & 1 & 0 \\\hline
10 & 1 & 0 & 1 & 0 & 0 \\\hline
11 & 1 & 0 & 1 & 1 & 0 \\\hline
12 & 1 & 1 & 0 & 0 & 0 \\\hline
13 & 1 & 1 & 0 & 1 & 1 \\\hline
14 & 1 & 1 & 1 & 0 & 1 \\\hline
15 & 1 & 1 & 1 & 1 & 0 \\\hline
\end{tabular}

\paragraph{DNF}
DNF = Minterme der einschlägigen Indizes (Sum of Products)\\
$m_2 + m_5 + m_6 + m_7 + m_{13} + m_{14} = \\
 \neg a.\neg b.c.\neg d + \neg a.b.\neg c.d + \neg a.b.c.\neg d + \neg a.b.c.d + a.b.\neg c.d + a.b.c.\neg d$

\paragraph{KNF}
KNF = Maxterme (=$\neg$ Minterme) der nullschlägigen Indizes (Product of Sums)\\
$M_0 . M_1 . M_3 . M_4 . M_8 . M_9 . M_{10} . M_{11} . M_{12} . M_{15}  = \\
(a+b+c+d) . (a+b+c+\neg d) . (a+b+\neg c+\neg d) . (a+\neg b+c+d) . (\neg a+b+c+d) . (\neg a+b+c+\neg d) . (\neg a+b+\neg c+d) . (\neg a+b+\neg c+\neg d) . (\neg a+\neg b+c+d) . (\neg a+\neg b+\neg c+\neg d)$


\end{document}