\documentclass[10pt,a4paper]{scrartcl}
\PassOptionsToPackage{table}{xcolor}
\usepackage[utf8]{inputenc}
\usepackage[T1]{fontenc}
\usepackage[ngerman]{babel}
\usepackage{microtype, multicol, marginnote, bera, parskip}
\usepackage{listings, amsmath, amssymb, graphicx, tikz, epic}
\usepackage{stmaryrd} %for lightning arrow
\usepackage{pstricks, pst-node, pst-tree, pdflscape}
\usepackage[babel=true]{csquotes}
\usepackage{placeins}
\tolerance=2000
\setcounter{secnumdepth}{0}
\usepackage[inner=2cm,outer=2cm,top=1.5cm,bottom=1.5cm,includeheadfoot]{geometry}
\usepackage{multirow}
\newcommand{\subExercise}[1]{\vspace{0.5em} \noindent{\bf #1)}}
\author{Michael Mardaus \and Andrey Tyukin}
\title{\includegraphics[scale=0.2]{../logo_schriftzug}\\
Technische Informatik: Abgabe 4}

\begin{document}

\maketitle

\section*{Exercise 4.1 (Full adder from decoder)}

\vspace{1em}
\begin{figure}[h]
  \centering\includegraphics[width=0.6\linewidth]{images/fullAdder.png}
\end{figure}
\vspace{1em}

\FloatBarrier
\newpage
\section*{Exercise 4.2 (Subtractors)}
\subExercise{a}
Here are the tables for the two circuits we wish to implement 
(namely Half-Subtractor and Full-Subtractor):

\vspace{0.5em}
\begin{tabular}{|c c|c c|}
  \hline
  minuend & subtrahend & underflow & difference \\
  \hline
  0 & 0 & 0 & 0 \\
  0 & 1 & 1 & 1 \\
  1 & 0 & 0 & 1 \\
  1 & 1 & 0 & 0 \\
  \hline
\end{tabular}

If we read $ms$ as numbers with high order bit on the left: 
\begin{align*}
  u_{out} &= m_1 \\
  d &= m_1 + m_2.
\end{align*}

\vspace{0.5em}
\begin{tabular}{|c c c|c c|}
  \hline
  minuend & subtrahend & underflow & underflow & difference \\
  \hline
  0 & 0 & 0 & 0 & 0 \\
  0 & 0 & 1 & 1 & 1 \\ 
  0 & 1 & 0 & 1 & 1 \\ 
  0 & 1 & 1 & 1 & 0 \\ 
  1 & 0 & 0 & 0 & 1 \\ 
  1 & 0 & 1 & 0 & 0 \\ 
  1 & 1 & 0 & 0 & 0 \\ 
  1 & 1 & 1 & 1 & 1 \\ 
  \hline
\end{tabular}

Again, in SOP-notation with high-order bit on the left:
\begin{align*}
  u_{out} &= m_1 + m_2 + m_3 + m_7 \\
  d &= m_1 + m_2 + m_4 + m_7.
\end{align*}

\subExercise{b} 
More or less compact symbolic representations 
of these two circuits are as follows 
(first component is always the resulting underflow, 
 second is the actual difference):
\[
  HalfSubtractor(m,s) = (\bar m s, m \not\leftrightarrow s)
\]
\[
  FullSubtractor(m,s,u) = 
    (\bar m \not\leftrightarrow su, 
     m \not\leftrightarrow s \not\leftrightarrow u
    )
\]

\subExercise{c} 
Now we want to simplify both components (difference and undeflow) 
of the full subtractor using Karnaugh diagrams. We begin with the
difference:

\vspace{0.5em}
\begin{tabular}{|c c|c c c c|}
  \hline 
    & & \multicolumn{4}{c|}{minuend / subtrahend} \\
    & & 00 & 01 & 11 & 10 \\
  \hline
    \multirow{2}{*}{underflow} & 0 &   & \cellcolor{blue}1 &   & \cellcolor{orange}1 \\
                               & 1 & \cellcolor{red}1 &   & \cellcolor{green}1 &   \\
  \hline
\end{tabular}

It seems that this diagram is not simplifiable at all: we have to cover every
one by an own 1x1 block. The simpliest expression for difference is thus:
\[
  d = \bar m \bar s u + \bar m s \bar u + m s u + m \bar s \bar u
\]

The ones for the output-underflow can be covered by
three 2x1 blocks, which all intersect at 011 (we use additive color combination,
light gray is supposed to be combination of red, green and blue):
\vspace{0.5em}
\begin{tabular}{|c c|c c c c|}
  \hline 
    & & \multicolumn{4}{c|}{minuend / subtrahend} \\
    & & 00 & 01 & 11 & 10 \\
  \hline
    \multirow{2}{*}{underflow} & 0 & 0 & \cellcolor{green}{1} & 0 & 0 \\
                               & 1 & \cellcolor{red}{1} & \cellcolor{lightgray}{1} & \cellcolor{blue}{1} & 0 \\
  \hline
\end{tabular}

Thus, the simplified formula for the output-underflow is:
\[
u_{out} = \bar m u + \bar m s + su.
\]

\subExercise{d}

\vspace{1em}
\begin{figure}[h]
  \centering\includegraphics[width=\linewidth]{images/halfSubtractor.png}
  \caption{Half subtractor.}
\end{figure}
\vspace{1em}

\vspace{1em}
\begin{figure}[h]
  \centering\includegraphics[width=\linewidth]{images/fullSubtractor.png}
  \caption{Full subtractor. We recycled as many gatters as we could for both sub-circuits. Reason: that's the maximal complexity allowed for free LucidChart accounts...}
\end{figure}
\vspace{1em}

\FloatBarrier
\section*{Exercise 4.3 (TODO)}
\subExercise{a}
\subExercise{b}

\end{document}
